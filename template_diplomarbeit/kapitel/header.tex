% Headerdatei der Diplomarbeit

\documentclass[
		pdftex,
		a4paper,
		12pt,
		twoside,
		openright,
		parskip,
		ngerman,
		chapterprefix,%       Kapitel anschreiben als Kapitel
]{scrreprt}

%Deutsche Trennungen, Anführungsstriche und mehr:
\usepackage[ngerman]{babel}

%Eingabe von ü,ä,ö,ß erlauben
\usepackage[utf8]{inputenc}
%Ausgabe von ä, ö, ü, ß erlauben
\usepackage[T1]{fontenc}

%Zum Einbinden von Grafiken
\usepackage{graphicx}

%Darstellung des Glossars einstellen
\usepackage[style=super, nonumberlist, toc=true]{glossaries}

\usepackage[%
	pdftitle={Der Beruf des Spürhundes in Gegenwart und Zukunft},%
	pdfauthor={Bjarne Friedjof Blue},%
	pdfcreator={LaTeX with hyperref and KOMA-Script},%
	pdfpagemode=UseOutlines,% Inhaltsverzeichnis anzeigen beim Öffnen
	pdflang=de%
]{hyperref}
\hypersetup{%
	colorlinks=true,%        Aktivieren von farbigen Links im Dokument (keine Rahmen)
	linkcolor=blue,%         Farbe festlegen.
	citecolor=blue,%         Farbe festlegen.
	filecolor=blue,%         Farbe festlegen.
	menucolor=blue,%         Farbe festlegen.
	urlcolor=blue,%          Farbe von URL's im Dokument.
	bookmarksnumbered=true%  Überschriftsnummerierung im PDF Inhalt anzeigen.
}

\makeglossaries
